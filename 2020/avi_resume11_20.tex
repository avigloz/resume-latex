% Resume (and underlying template) (C) Avi Glozman, 2019-2020.
% See LICENSE for more information.

\documentclass[10pt]{article}
\usepackage[a4paper,  margin=0.5in, top=0.5in, bottom=0.5in]{geometry}
\usepackage{hyperref}

\usepackage[T1]{fontenc}
\usepackage{lmodern}

\renewcommand*\familydefault{\sfdefault} %% Only if the base font of the document is to be sans serif

\usepackage[many]{tcolorbox}
\usetikzlibrary{decorations.pathreplacing}

% Source: https://tex.stackexchange.com/a/277825
% I also include some personal customization.
\newtcolorbox{leftli}{
    enhanced jigsaw, 
    breakable, % allow page breaks
    frame hidden, % hide the default frame
	colback=gray!15,
    overlay={%
        \draw [
            fill=none, % fill paper
			decoration={brace,amplitude=0},
            decorate,
            thin,
            black
        ]
        % right line
        (frame.south west)--(frame.north west);
    },
    % paragraph skips obeyed within tcolorbox
    parbox=false,
}

\begin{document}
	\pagestyle{empty}
	\begin{center}
		{\huge \textbf{Avi Glozman}} {\large \hfill 425.802.6718 --- \href{mailto:avi@avigloz.net}{avi@avigloz.net}}\\
		\vspace{1.25mm}
		{\large American and Israeli Dual Citizen \hfill \href{https://avigloz.net}{avigloz.net} --- \href{https://github.com/avigloz}{github.com/avigloz}}
	\end{center}
	
	\begin{flushleft}	
		\vspace{-1.65mm}
		{\large \raggedright \textbf{Education}}
		\vspace{1.25mm}
	
		\hrule
		
		\vspace{2.25mm}
		\textbf{University of Pittsburgh}, Pittsburgh, PA\\
      	{\small B.Sc. in Computer Science \hfill \textit{August 2018 --- December 2020 (Expected)}}\\
		{\small Dean's List: \textit{Spring 2020}}\\
		{\small \textbf{Notable coursework:} Web Applications, Data Structures, Algorithm Implementation, Intro OS, Intro Systems Software (C), Computer Organization and Assembly (MIPS), Discrete Mathematics, Formal Methods}
	
		\vspace{1.25mm} 
		{\large \raggedright \textbf{Skills}}
		\vspace{1.25mm}
	
		\hrule
	
		\vspace{2.25mm}
		\textbf{Programming languages:} C, C\verb!#!, C\texttt{++}, Java, Node.js, Python\\
		\vspace{0.5mm}
		\textbf{Concepts:} REST, Sockets, Distributed Systems, Machine Learning, Operating Systems\\
		\vspace{0.5mm}
		\textbf{Software:} Git, Linux, Windows, SQL, NoSQL, Azure, \LaTeX\\ 
		\vspace{0.5mm}
		\textbf{Spoken languages:} English (Native), Russian (Native), Spanish (Partial fluency),  Hebrew (Basic), Mandarin (Very basic)

		\vspace{1.5mm}
		{\large \raggedright \textbf{Professional Experience}}
		\vspace{1.25mm}
	
		\hrule

		\vspace{2.25mm}
		\textbf{University of Pittsburgh, SCI}, Pittsburgh, PA\\
		\begin{leftli}
         {\small \textbf{Machine Learning Researcher} \textit{(Capstone Project)}}  \hfill \textit{\small May 2020 --- September 2020}
			\begin{itemize}
				\item Trained a YOLOv3 object detection model using Darknet for detecting parts of the human spinal anatomy with up to 98\verb!%! accuracy
				\vspace{-2mm}
				\item Developed a pipeline for converting CT scan data (DICOM format) into a 3D model, then into augmented images for use as a synthetic dataset for model training and testing
			\end{itemize}
		\end{leftli}

		\begin{leftli}
			{\small \textbf{Undergraduate Researcher, Learning Technologies Lab}} \hfill \textit{\small November 2019 --- Present}
			\begin{itemize}
				\item Using Python and BeautifulSoup to efficiently scrape hundreds of faculty profiles to gain various insights, such as specific research interests, publication data, lab affiliations, etc.
				\vspace{-2mm}
				\item Compiling scraped data into easy-to-process datasets for use in a university-wide undergraduate research opportunity discovery platform
			\end{itemize}
		\end{leftli}

		\vspace{-1.50mm}
		\textbf{Uber}, Pittsburgh, PA\\
		\begin{leftli}
			{\small \textbf{Software Engineering Intern, Advanced Technologies Group (ATG), Simulation}} \hfill \textit{\small May 2019 --- August 2019}
			\begin{itemize}
				\item Created an ETL system using Python for moving self-driving car related data from DynamoDB to PostgreSQL, and solved complex data-syncing challenges
				\vspace{-2mm}
				\item Developed pruning algorithms in Python for preventing transfer of broken, invalid, and\texttt{/}or redundant data relating to self-driving car testing
				\vspace{-2mm}
				\item Contributed significantly to a web API written in Go for self-driving car data analysis in production
			\end{itemize}
		\end{leftli}

		\vspace{-1.50mm}
		\textbf{aspace}, Seattle, WA\\
		\begin{leftli}
			{\small \textbf{Lead Software Engineer, Backend}} \hfill \textit{\small May 2017 --- October 2017 (11th-12th grade)}

			\begin{itemize}
				\item Designed MySQL and MongoDB databases for storing parking spot sensor data and user data, respectively
				\vspace{-2mm}
				\item Designed a RESTful API written in Node.js to support UX on Android and iOS apps, and for recieving sensor data
				\vspace{-2mm}
				\item Wrote an implementation of Dijkstra's algorithm using Node.js for navigation, relying on user location data and data from Mapbox's API
				\vspace{-2mm}
				\item Used Twilio's SMS API to integrate two-factor authentication into the backend.
			\end{itemize}
		\end{leftli}

		{\footnotesize \textit{Please note that the present lack of software engineering internships is solely a consequence of my quick completion of my degree.}}

		\vspace{1.25mm}
		{\large \raggedright \textbf{Noteworthy Technical Projects}}
		\vspace{1.25mm}
	
		\hrule
	
		\vspace{2.25mm}
		\textbf{Virtual Memory Simulator} \textit{(schoolwork)} \hfill \textit{\small 2020}
		\vspace{-2mm}
		\begin{itemize}
			\item Wrote a small virtual memory simulator in Java, specifically for demonstrating the Second Chance local page replacement algorithm
			\vspace{-2mm}
			\item Supports both 32 and 64-bit virtual memory addresses, and outputs cumulative results such as \verb!#! of page faults, memory accesses, and writes to disk per process
		\end{itemize}
		\vspace{-1.5mm}
		\pagebreak
		\textbf{Semaphores and Synchronization} \textit{(schoolwork)} \hfill \textit{\small 2020}
		\vspace{-2mm}
		\begin{itemize}
			\item Modified the Linux 2.6.23 kernel to add a custom semaphore implementation, with testing done using QEMU
			\vspace{-2mm}
			\item Added new syscalls for synchronization/process scheduling using FIFO
		\end{itemize}
		\vspace{-1.5mm}
		\textbf{Crossword Solver} \textit{(schoolwork)} \hfill \textit{\small 2020}
		\vspace{-2mm}
		\begin{itemize}
			\item Created a fully functional NxN crossword puzzle solver, using a recursive backtracking algorithm and heuristics for verifying solutions
		\end{itemize}
		\vspace{-1.5mm}
		\textbf{Asteroids} \textit{(schoolwork)} \hfill \textit{\small 2020}
		\vspace{-2mm}
		\begin{itemize}
			\item Created a fully functional Asteroids clone in MIPS assembly, using a provided graphics engine and the MARS IDE
			\vspace{-2mm}
			\item Implemented physics/movement, random asteroid generation, and collision logic from scratch.
		\end{itemize}
		\vspace{-1.5mm}
		\textbf{Lightweight Messenger} \textit{(schoolwork)} \hfill \textit{\small 2020}
		\vspace{-2mm}
		\begin{itemize}
			\item Used Node.js and Socket.io to design a backend for a lightweight messaging system prototype
			\vspace{-2mm}
			\item Implemented a RESTful API with Express.js for interacting with a PostgreSQL database for user accounts, contacts, message history, etc.
		\end{itemize}
		\vspace{-1.5mm}
		\textbf{\href{https://github.com/avigloz/synesthesia}{Synesthesia}} \textit{(personal, open source via GitHub)} \hfill \textit{\small 2020 --- Present}
		\vspace{-2mm}
		\begin{itemize}
			\item Program that creates unique patterns/visualizations for any sort of audio, by algorithmically interpreting and "displaying" the sound in a visually satisfying manner (using C\texttt{++})
		\end{itemize}
		\vspace{-1.5mm}
		\textbf{\href{https://rentnexus.net}{rentnexus.net}} \textit{(personal)} \hfill \textit{\small 2019 --- Present}
		\vspace{-2mm}
		\begin{itemize}
			\item Site that provides a focused platform for \textit{students} to post about housing opportunities and requests, e.g sublet notices, roommate requests, etc. around campus
		\end{itemize}

		\vspace{1.5mm}
		{\large \raggedright \textbf{Extracurriculars}}
		\vspace{1.25mm}
	
		\hrule
	
		\vspace{2.25mm}
		Pitt Computer Science Club (CSC) Member and Mentor\hfill \textit{\small September 2018 --- Present}\\
		%Private Online Mathematics and CS Tutor \hfill \textit{October 2017 --- Present}
		Wikipedia Contributor (under username \textit{Avigl}) --- \href{https://pageviews.toolforge.org/?project=en.wikipedia.org&platform=all-access&agent=user&redirects=0&range=all-time&pages=Timeline_of_social_media|Timeline_of_online_advertising|Timeline_of_e-commerce|Screening_Partnership_Program|Silicon_Valley_Education_Foundation|Chicago_Community_Trust}{over 700k all-time pageviews}\hfill \textit{\small June 2016 --- Present}
		
	\end{flushleft}
\end{document}
